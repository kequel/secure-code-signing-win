\documentclass{article}
\usepackage{graphicx}
\usepackage[T1]{fontenc}
\usepackage[polish]{babel} 

\title{Podpisanie i weryfikacja kodu aplikacji w systemie Windows}
\author{Martyna Borkowska, Karolina Glaza, Amila Amarasekara}
\date{Maj 2025}

\begin{document}

\maketitle

\section*{Wstęp teorytyczny}
\subsection*{Podpis cyfrowy aplikacji}
sposób zapewnienia, że:
\begin{itemize}
    \item kod pochodzi od zaufanego wydawcy (autentyczność)
    \item nie został zmodyfikowany po podpisaniu (integralność)
    \item może być zaufany przez system operacyjny 
\end{itemize}
Systemy operacyjne coraz częściej wymagają, by aplikacje były podpisane – szczególnie przy dystrybucji przez internet. Nieużycie podpisu może skutkować:
\begin{itemize}
    \item ostrzeżeniami bezpieczeństwa ("Nieznany wydawca")
    \item blokadą instalacji przez system
\end{itemize}

\subsection*{Działanie}
1. Wydawca aplikacji generuje parę kluczy: prywatny i publiczny.\\
2. Certyfikat X.509 (np. od CA jak DigiCert, Sectigo) zawiera klucz publiczny i dane właściciela.\\
3. Kod jest podpisywany kluczem prywatnym –  hash + podpis.\\
4. System używa klucza publicznego z certyfikatu do weryfikacji podpisu.

\subsection*{Timestamp (znacznik czasu)}
To dowód, że podpis był ważny w momencie jego złożenia, nawet jeśli certyfikat wygasł później.\\

Wymaga połączenia z usługą TSA (Time Stamping Authority), np.:
\begin{itemize}
    \item http://timestamp.digicert.com
    \item http://timestamp.sectigo.com
\end{itemize}

\section*{Krok 1: Generowanie własnego Urzędu Certyfikacji (CA)}

Na potrzeby projektu utworzono własny Urząd Certyfikacji (CA) za pomocą narzędzia OpenSSL uruchamianego w PowerShellu jako administrator 2 wyniku którego wygenerowano:
\begin{itemize}
  \item \texttt{ca.key} – klucz prywatny CA (należy chronić jak hasło),
  \item \texttt{ca.crt} – certyfikat CA do dystrybucji.
\end{itemize}
Użyto polecenia:

\begin{verbatim}
openssl req -x509 -newkey rsa:2048 -keyout ca.key -out ca.crt
-days 365 -addext "basicConstraints=CA:TRUE" 
\end{verbatim}

\section*{Krok 2: Dodanie CA do zaufanych certyfikatów w Windows}

Aby system Windows ufał certyfikatowi aplikacji, należało dodać certyfikat CA do magazynu zaufanych głównych urzędów certyfikacji. W tym celu użyto narzędzia \texttt{certmgr.msc}, wybierając opcję „Importuj...” i wskazując plik \texttt{ca.crt}. Po zakończeniu importu, podpisy wydawane przez ten CA będą traktowane jako zaufane.

\section*{Krok 3: Generowanie certyfikatu dla aplikacji}

Wygenerowano parę kluczy dla aplikacji oraz żądanie certyfikatu (CSR), następnie podpisano CSR za pomocą CA. W wyniku czego powstały pliki:
\begin{itemize}
  \item \texttt{app.key} – klucz prywatny aplikacji,
  \item \texttt{app.crt} – certyfikat aplikacji podpisany przez CA,
  \item \texttt{ca.srl} – plik numerów seryjnych certyfikatów CA.
\end{itemize}

\begin{verbatim}
openssl req -newkey rsa:2048 -keyout app.key -out app.csr -nodes
\end{verbatim}

\begin{verbatim}
openssl x509 -req -in app.csr -CA ca.crt -CAkey ca.key 
-CAcreateserial -out app.crt -days 365 -extfile v3.ext
\end{verbatim}


\section*{Krok 4: Podpisywanie kodu w Visual Studio}

Do podpisywania kodu potrzebny był plik \texttt{.pfx}, łączący klucz prywatny i certyfikat. W Visual Studio, w ustawieniach projektu, wybrano plik \texttt{app.pfx} do podpisywania. Visual Studio oznacza aplikację jako podpisaną, ale Eksplorator Windows może nie pokazać podpisu, jeśli CA nie jest ogólnie zaufane.Generacja pliku:

\begin{verbatim}
openssl pkcs12 -export -out app.pfx -inkey app.key -in app.crt
\end{verbatim}

\section*{Krok 5: Ręczne podpisywanie kodu z timestampem}

W celu poprawnego podpisania aplikacji ręcznie użyto narzędzia \texttt{signtool.exe}, które umożliwia dołączenie znacznika czasu (timestamp). Dzięki dodaniu znacznika czasu podpis pozostanie ważny nawet po wygaśnięciu certyfikatu.

\begin{verbatim}
signtool sign /f "C:\WINDOWS\System32\app.pfx" /p silnehaslo
/fd SHA256 /tr http://timestamp.digicert.com /td SHA256
"MKA_Demo_App.exe"
\end{verbatim}

\section*{Krok 6: Weryfikacja podpisu}

Sprawdzono poprawność podpisu poprzez właściwości pliku w Eksploratorze Windows (zakładka „Podpisy cyfrowe”) oraz za pomocą:

\begin{verbatim}
signtool verify /v /pa "MKA_Demo_App.exe"
\end{verbatim}

\section*{Podsumowanie}
 W ramach projektu zrealizowano proces pełnego podpisywania kodu aplikacji w systemie Windows. Wykonane kroki:
\begin{itemize}
    \item Certyfikaty:\\
Wygenerowano własny Urząd Certyfikacji (CA) oraz certyfikat aplikacji z rozszerzeniami v3 i dodano CA do zaufanych  w systemie Windows, aby umożliwić weryfikację podpisów.
    \item Podpisywanie kodu:\\
Skonfigurowano podpisywanie projektu w IDE (Visual Studio) z użyciem certyfikatu .pfx i przeprowadzono ręczne podpisywanie aplikacji przy pomocy signtool.exe.
    \item Timestamp:\\
Zastosowano usługę znakowania czasem (TSA) w celu zapewnienia ważności podpisu mimo ewentualnego wygaśnięcia certyfikatu.
    \item Weryfikacja:\\
Sprawdzono ważność i autentyczność podpisu z poziomu eksploratora Windows oraz narzędzia signtool verify.
\end{itemize}

\end{document}
