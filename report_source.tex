\documentclass{article}
\usepackage{graphicx}
\usepackage[T1]{fontenc}
\usepackage[polish]{babel} 
\usepackage{geometry}
\usepackage{listings}
\usepackage{xcolor}
\usepackage{float}
\usepackage{courier}
\usepackage{url}
\geometry{a4paper, margin=2.5cm}

\lstdefinestyle{command}{
    basicstyle=\small\ttfamily\color{black},
    backgroundcolor=\color{gray!10},
    frame=single,
    rulecolor=\color{gray},
    tabsize=4,
    captionpos=b,
    breaklines=true,
    breakatwhitespace=true,
    numbers=none,
    showspaces=false,
    showstringspaces=false,
    showtabs=false
}

\title{Podpisanie i weryfikacja kodu aplikacji w systemie Windows}
\author{Martyna Borkowska, Karolina Glaza, Amila Amarasekara \\ \url{https://github.com/kequel}}
\date{Maj 2025}

\begin{document}

\maketitle

\section*{Zagadnienia}
\textbf{1.} Demonstracja certyfikatu wystawionego przez dowolne CA z właściwymi rozszerzeniami (v3 Extensions) oraz zaufania do CA, który go wystawił na testowej maszynie. \\ \\
\textbf{2.} Demonstracja podpisywania kodu w dowolnym IDE (przykładowo VisualStudio, Eclipse) - demonstracja weryfikacji podpisu. \\ \\
\textbf{3.}Demonstracja podpisywania kodu ręcznie w oparciu o dowolne narzędzie.\\ \\
\textbf{4.}Demonstracja ręcznego podpisywania kodu z użyciem publicznie dostępnej usługi znakowania czasem (podpisu odpornego na nieaktualność certyfikatu). \\



\section*{Wstęp teorytyczny}
\subsection*{Podpis cyfrowy aplikacji}
sposób zapewnienia, że:
\begin{itemize}
    \item kod pochodzi od zaufanego wydawcy (autentyczność)
    \item nie został zmodyfikowany po podpisaniu (integralność)
    \item może być zaufany przez system operacyjny 
\end{itemize}
Systemy operacyjne coraz częściej wymagają, by aplikacje były podpisane - szczególnie przy dystrybucji przez internet. Nieużycie podpisu może skutkować:
\begin{itemize}
    \item ostrzeżeniami bezpieczeństwa ("Nieznany wydawca")
    \item blokadą instalacji przez system
\end{itemize}

\subsection*{Działanie}
1. Wydawca aplikacji generuje parę kluczy: prywatny i publiczny.\\
2. Certyfikat X.509 (np. od CA jak DigiCert, Sectigo) zawiera klucz publiczny i dane właściciela.\\
3. Kod jest podpisywany kluczem prywatnym -  hash + podpis.\\
4. System używa klucza publicznego z certyfikatu do weryfikacji podpisu.

\subsection*{Timestamp (znacznik czasu)}
To dowód, że podpis był ważny w momencie jego złożenia, nawet jeśli certyfikat wygasł później.\\

Wymaga połączenia z usługą TSA (Time Stamping Authority), np.:
\begin{itemize}
    \item http://timestamp.digicert.com
    \item http://timestamp.sectigo.com
\end{itemize}



\section{Generowanie własnego Urzędu Certyfikacji (CA)}
Generowanie własnego Urzędu Certyfikacji (CA) to proces tworzenia zaufanego podmiotu, który będzie wystawiał certyfikaty dla aplikacji. \\
\subsection{Wymagane narzędzia: }
\begin{itemize}
    \item \textbf{OpenSSL} działające w terminalu – narzędzie kryptograficzne,
    \item \textbf{PowerShell z uprawnieniami administratora} - do wykonania komend systemowych,
    \item \textbf{Plik konfiguracyjny} (opcjonalnie) – dla zaawansowanych ustawień CA. \\
\end{itemize}

\subsection{Procedura generowania CA}
\begin{enumerate}
    \item Uruchom PowerShell jako administrator.
    \item Wykonaj komendę generującą klucz i certyfikat CA:
\end{enumerate}

\begin{lstlisting}[style=command, caption=Generowanie CA w PowerShell]
openssl req -x509 -newkey rsa:2048 \
    -keyout ca.key -out ca.crt \
    -days 365 \
    -addext "basicConstraints=CA:TRUE"
\end{lstlisting}

\noindent\textbf{\\Parametry:}
\begin{itemize}
    \item \texttt{x509} - generuje certyfikat samopodpisany (X.509),
    \item \texttt{newkey rsa:2048} - tworzy klucz RSA 2048-bitowy,
    \item \texttt{days 365} - okres ważności certyfikatu,
    \item \texttt{addext} - dodaje rozszerzenia krytyczne dla CA
\end{itemize}

\begin{figure}[H]
    \centering
    \includegraphics[width=0.9\textwidth]{screenshots/gen_ca_1}
    \caption{Uzupełnione dane.}
    \label{fig:gen_ca}
\end{figure}

\subsection{Wygenerowane pliki}
\begin{figure}[H]
    \centering
    \includegraphics[width=0.7\textwidth]{screenshots/check_ca_1}
    \caption{Struktura plików CA po generacji}
    \label{fig:ca_files}
\end{figure}
Po wykonaniu komendy powstają: \\
\begin{itemize}
    \item \texttt{ca.crt} - certyfikat publiczny CA:
    \begin{itemize}
        \item Zawiera klucz publiczny i metadane,
        \item Może być dystrybuowany do zaufanych magazynów. \\
    \end{itemize}
    \item \texttt{ca.key} - klucz prywatny CA. 
Jeżeli zaczyna się od:
\begin{itemize}
\item \textbf{- - - - -BEGIN ENCRYPTED PRIVATE KEY- - - -} - klucz jest zaszyfrowany
\item \textbf{- - - - -BEGIN PRIVATE KEY- - - -} - nie ma ochrony hasłem
\end{itemize}

    \begin{figure}[H]
             \centering
                \includegraphics[width=0.7\textwidth]{screenshots/ca_key_1}
            \caption{Plik ca.key.}
        \label{fig:ca_files}
    \end{figure}
    

\end{itemize}


\section{Dodanie CA do zaufanych certyfikatów w Windows}
Aby system Windows ufał certyfikatom podpisanym przez nasze CA, konieczne jest dodanie certyfikatu \texttt{ca.crt} do magazynu \textbf{Zaufane główne urzędy certyfikacji}. 
W tym celu należy uruchomić \textbf{certmgr.msc} – narzędzie MMC do zarządzania certyfikatami użytkownika, z uprawnieniami administratora.

\subsection{Procedura importu CA}
\begin{enumerate}
    \item Otwórz \texttt{certmgr.msc} poprzez:
    \begin{itemize}
        \item Menu Start $\rightarrow$ "Zarządzanie certyfikatami", lub
        \item Uruchom $\rightarrow$ \texttt{certmgr.msc}
    \end{itemize}

    \item W drzewie nawigacyjnym wybierz:
    \begin{lstlisting}[]
Certyfikaty -> Biezacy uzytkownik 
    -> Zaufane glowne urzedy certyfikacji 
    -> Certyfikaty
    \end{lstlisting}

    \begin{figure}[H]
    \centering
    \includegraphics[width=0.85\textwidth]{screenshots/cert_2}
    \caption{Interfejs certmgr.msc.}
    \label{fig:certmgr_ui}
\end{figure}

    \item Kliknij prawym przyciskiem i wybierz \textbf{Wszystkie zadania $\rightarrow$ Importuj...}
\\
    \item W kreatorze importu:
    \begin{itemize}
        \item Wskaż plik \texttt{ca.crt},
        \item Wybierz magazyn \textbf{Zaufane główne urzędy certyfikacji},
        \item Potwierdź operację. \\ \\ \\ \\ 
    \end{itemize}
\end{enumerate}

\begin{figure}[h]
    \centering
    \includegraphics[width=0.85\textwidth]{screenshots/choose_2}
    \caption{Wybór ca.crt w certmgr.msc.}
    \label{fig:certmgr_ui}
\end{figure}

\subsection{Weryfikacja poprawności importu}
W narzędziu sprawdź, czy certyfikat pojawił się na liście zaufanych CA:
\begin{figure}[H]
    \centering
    \includegraphics[width=0.8\textwidth]{screenshots/list_2}
    \caption{Lista zaufanych certyfikatów z widocznym certyfikatem "MKA CA".}
    \label{fig:ca_in_store}
\end{figure}

\subsection{Konsekwencje}
\begin{itemize}
    \item \textbf{System będzie ufał wszystkim certyfikatom} podpisanym przez ten CA,
    \item Klucz \texttt{ca.key} musi być chroniony - jego kompromitacja oznacza utratę zaufania do całej infrastruktury,
\end{itemize}
\section{Generowanie certyfikatu dla aplikacji}

Aby możliwe było podpisanie aplikacji, należy wygenerować parę kluczy oraz certyfikat podpisany przez wcześniej utworzony CA. Proces ten składa się z trzech głównych kroków: wygenerowania żądania certyfikatu (CSR), jego podpisania oraz konfiguracji rozszerzeń X.509. \\

\subsection{Generowanie klucza i żądania CSR}

W pierwszym kroku generujemy klucz prywatny aplikacji oraz żądanie certyfikatu (CSR):

\begin{lstlisting}[style=command, caption=Generowanie klucza i CSR]
openssl req -newkey rsa:2048 -keyout app.key -out app.csr -nodes
\end{lstlisting}

\noindent\textbf{Opis parametrów:}
\begin{itemize}
    \item \texttt{newkey rsa:2048} - generuje nowy klucz RSA o długości 2048 bitów,
    \item \texttt{keyout app.key} - zapisuje klucz prywatny do pliku,
    \item \texttt{out app.csr} - tworzy żądanie certyfikatu (CSR),
    \item \texttt{nodes} - nie szyfruje klucza prywatnego (brak hasła).
\end{itemize}

\begin{figure}[H]
    \centering
    \includegraphics[width=0.9\textwidth]{screenshots/gen_cert_3}
    \caption{Wprowadzenie danych do CSR}
    \label{fig:cert_dn}
\end{figure}

\noindent W wyniku wykonania tej komendy powstają dwa pliki:
\begin{itemize}
    \item \texttt{app.key} - zawiera klucz prywatny aplikacji. Jest to klucz, którym będzie podpisywany kod. Należy chronić go przed nieautoryzowanym dostępem, ponieważ jego utrata lub wyciek może zagrozić bezpieczeństwu całego procesu podpisywania.
    \item \texttt{app.csr} - żądanie podpisania certyfikatu (Certificate Signing Request). Zawiera dane identyfikacyjne (np. nazwa organizacji, CN, kraj) oraz klucz publiczny odpowiadający prywatnemu z \texttt{app.key}. Plik ten jest przeznaczony do przekazania urzędowi certyfikacji (CA) w celu wystawienia certyfikatu.
\end{itemize}


\begin{figure}[H]
    \centering
    \includegraphics[width=0.8\textwidth]{screenshots/check_3}
    \caption{Wygenerowane pliki: \texttt{app.key} i \texttt{app.csr}}
    \label{fig:key_csr_files}
\end{figure}

\subsection{Podpisanie CSR przez CA}

Następnie podpisujemy CSR przy użyciu wcześniej wygenerowanego urzędu CA:

\begin{lstlisting}[style=command, caption=Podpisywanie CSR przez CA]
openssl x509 -req -in app.csr \
    -CA ca.crt -CAkey ca.key \
    -CAcreateserial \
    -out app.crt \
    -days 365 \
    -extfile v3.ext
\end{lstlisting}

\noindent\textbf{Opis parametrów:}
\begin{itemize}
    \item \texttt{req -in app.csr} - określa CSR do podpisania,
    \item \texttt{CA ca.crt -CAkey ca.key} - pliki CA używane do podpisu,
    \item \texttt{CAcreateserial} - tworzy plik z numerem seryjnym,
    \item \texttt{out app.crt} - wynikowy certyfikat aplikacji,
    \item \texttt{days 365} - ważność certyfikatu (w dniach),
    \item \texttt{extfile v3.ext} - plik z rozszerzeniami certyfikatu.
\end{itemize}

\begin{figure}[H]
    \centering
    \includegraphics[width=0.9\textwidth]{screenshots/request_3}
    \caption{Proces podpisywania certyfikatu}
    \label{fig:signing_process}
\end{figure}

\subsection{Plik konfiguracyjny \texttt{v3.ext}}

Podczas podpisywania żądania CSR konieczne jest podanie pliku z rozszerzeniami X.509v3, które określają dodatkowe właściwości certyfikatu. Taki plik nazywa się najczęściej \texttt{v3.ext} i zawiera zestaw dyrektyw opisujących, jak systemy powinny interpretować dany certyfikat.

\noindent Przykładowa zawartość pliku \texttt{v3.ext}:

\begin{figure}[H]
    \centering
    \includegraphics[width=0.6\textwidth]{screenshots/v3_3}
    \caption{Zawartość pliku \texttt{v3.ext}}
    \label{fig:v3_content}
\end{figure}

\noindent\textbf{Opis kluczowych rozszerzeń:}
\begin{itemize}
    \item \texttt{extendedKeyUsage = codeSigning} - informuje, że certyfikat może być używany do podpisywania kodu źródłowego lub wykonywalnego. Jest to wymagane przez narzędzia takie jak \texttt{signtool}.
    \item \texttt{subjectKeyIdentifier = hash} - dodaje identyfikator klucza publicznego w postaci skrótu, co ułatwia zarządzanie i sprawdzanie powiązań w łańcuchu certyfikatów.
    \item \texttt{authorityKeyIdentifier = keyid,issuer} - wskazuje na certyfikat nadrzędny (czyli CA), który wydał ten certyfikat. Umożliwia prawidłową budowę i walidację łańcucha zaufania.
\end{itemize}

\noindent Tego typu plik jest standardem w procesie tworzenia certyfikatów X.509 i zapewnia, że certyfikat zostanie prawidłowo rozpoznany przez oprogramowanie systemowe, przeglądarki, IDE i inne narzędzia korzystające z infrastruktury klucza publicznego. \\

\subsection{Wynik operacji}

Po wykonaniu wszystkich kroków, otrzymujemy zestaw plików gotowych do użycia w procesie podpisywania aplikacji:

\begin{itemize}
    \item \texttt{app.key} - klucz prywatny aplikacji,
    \item \texttt{app.crt}  certyfikat podpisany przez CA,
    \item \texttt{ca.srl} - numer seryjny podpisanego certyfikatu.
\end{itemize}

\begin{figure}[H]
    \centering
    \includegraphics[width=0.8\textwidth]{screenshots/second_check_3}
    \caption{Ostateczne pliki: certyfikat aplikacji i pliki CA}
    \label{fig:final_files}
\end{figure}

{\vskip 4\baselineskip}

\section{Podpisywanie kodu w Visual Studio}

\subsection{Przygotowanie certyfikatu PFX}
Plik PFX (\texttt{.pfx}) jest niezbędny do podpisywania aplikacji w środowiskach takich jak Visual Studio. Format ten łączy w jednym pliku certyfikat (\texttt{.crt}) oraz odpowiadający mu klucz prywatny (\texttt{.key}), opcjonalnie chroniąc je hasłem. Do jego utworzenia służy polecenie:

\begin{lstlisting}[style=command, caption=Generowanie PFX]
openssl pkcs12 -export -out app.pfx -inkey app.key -in app.crt
\end{lstlisting}

\noindent\textbf{Opis parametrów:}
\begin{itemize}
    \item \texttt{export} - tryb eksportu: tworzy nowy plik PFX,
    \item \texttt{out app.pfx} - nazwa wynikowego pliku PFX,
    \item \texttt{inkey app.key} - wskazuje klucz prywatny aplikacji,
    \item \texttt{in app.crt} - wskazuje certyfikat, który zostanie dołączony.
\end{itemize}

\begin{figure}[H]
    \centering
    \includegraphics[width=0.9\textwidth]{screenshots/pfx_check_4}
    \caption{Generowanie pliku PFX w PowerShell oraz wygenerowany plik.}
    \label{fig:pfx_export}
\end{figure}

\subsection{Próba podpisania w Visual Studio}
\begin{figure}[H]
    \centering
    \includegraphics[width=0.85\textwidth]{screenshots/first_visual_4}
    \caption{Konfiguracja podpisywania w zakładce właściwości.}
    \label{fig:vs_signing}
\end{figure}
\begin{figure}[H]
    \centering
    \includegraphics[width=0.85\textwidth]{screenshots/second_visual_4}
    \caption{Po podpisaniu plikem app.pfx.}
    \label{fig:vs_signing}
\end{figure}

Mimo, że wesług Visual Studio podpisany, w właściwościach aplikacji w bin/debug podpis nie będzie widoczny. 
Eksplorator plików pokazuje podpisy tylko od zaufanych urzędów certyfikacji (np. DigiCert). 
Certyfikat self-signed może być uznany za "nieznany", należy więc przeprowadzić ręczne podpisywanie. \\ \\ 

\section{Ręczne podpisywanie kodu z timestampem}

\subsection{Proces podpisywania signtool}

Do ręcznego podpisywania aplikacji w systemie Windows wykorzystujemy narzędzie \texttt{signtool.exe}, które jest częścią zestawu Windows SDK. Poniższa komenda podpisuje plik wykonywalny przy użyciu wcześniej wygenerowanego certyfikatu w formacie PFX oraz dodaje znacznik czasu z serwera TSA.

\begin{lstlisting}[style=command, caption=Podpisywanie signtool]
signtool sign /f "C:\WINDOWS\System32\app.pfx" /p silnehaslo 
/fd SHA256 /tr http://timestamp.digicert.com /td SHA256 
"MKA_Demo_App.exe"
\end{lstlisting}

\noindent\textbf{Opis parametrów:}
\begin{itemize}
    \item \texttt{sign} - polecenie podpisania pliku,
    \item \texttt{/f} - ścieżka do pliku PFX zawierającego certyfikat i klucz prywatny,
    \item \texttt{/p} - hasło do pliku PFX,
    \item \texttt{/fd SHA256} – algorytm skrótu (hashowania) używany do tworzenia podpisu,
    \item \texttt{/tr http://timestamp.digicert.com} - adres serwera TSA (Time Stamping Authority), który dostarcza znacznik czasu,
    \item \texttt{/td SHA256} - algorytm skrótu używany przy żądaniu znacznika czasu,
    \item \texttt{MKA\_Demo\_App.exe} - podpisywany plik wykonywalny.
\end{itemize}

\noindent Zastosowanie znacznika czasu (\texttt{/tr}) jest kluczowe - dzięki niemu podpis zachowuje ważność nawet po wygaśnięciu certyfikatu. Algorytm SHA256 zapewnia nowoczesny poziom bezpieczeństwa i jest zgodny z aktualnymi wymaganiami Microsoftu.


\begin{figure}[H]
    \centering
    \includegraphics[width=0.9\textwidth]{screenshots/signtool_4}
    \caption{Ręczne podpisywanie aplikacji.}
    \label{fig:manual_signing}
\end{figure}

\subsection{Weryfikacja podpisu}
Sprawdzono poprawność podpisu poprzez właściwości pliku w Eksploratorze Windows (zakładka „Podpisy cyfrowe”) oraz za pomocą:

\begin{lstlisting}[style=command, caption=Weryfikacja podpisu]
signtool verify /v /pa "MKA_Demo_App.exe"
\end{lstlisting}


\begin{figure}[H]
    \centering
    \includegraphics[width=0.9\textwidth]{screenshots/signtool_check_4}
    \caption{Weryfikacja podpisu w PowerShell.}
    \label{fig:signature_check}
\end{figure}

\begin{figure}[H]
    \centering
    \includegraphics[width=0.35\textwidth]{screenshots/propreties_after_signtool_4}
    \caption{Weryfikacja podpisu cyfrowego w oknie właściwości pliku.}
    \label{fig:digital_signature}
\end{figure}

\subsection{Techniczne aspekty timestampingu}
\begin{itemize}
    \item Serwer TSA: \texttt{http://timestamp.digicert.com}
    \item Algorytm hashowania: SHA256
    \item Weryfikacja łańcucha certyfikatów TSA \\
\end{itemize}

\section{Skrypt automatyzujący proces}
W ramach projektu zdecydowaliśmy się napisać także krótki skrypt automatyzujący cały powyższy proces podpisywania kodu, realizując wszystkie kluczowe kroki opisane w projekcie:

\begin{itemize}
    \item Generację własnego Urzędu Certyfikacji (CA)
    \item Dodawanie CA do zaufanych certyfikatów w Windows
    \item Tworzenie certyfikatu dla aplikacji
    \item Generację pliku PFX
    \item Podpisywanie plików wykonywalnych z timestampem
    \item Weryfikację poprawności podpisu \\
\end{itemize}

\subsection{Parametry skryptu (wymagają uzupełnienia odpowiednimi danymi):}

\begin{lstlisting}[]
$caName = "MKA2 CA"                # Nazwa urzedu certyfikacji
$appName = "MKA_App2"              # Nazwa aplikacji
$password = "silnehaslo"           # Haslo do certyfikatow
$exePath = "sciezka/do/aplikacji"  # Sciezka do pliku .exe
$daysValid = 365                   # Waznosc certyfikatow
$timestampServer = "http://timestamp.digicert.com"
\end{lstlisting}

\subsection{Uruchomienie skryptu}
\begin{lstlisting}[style=command, caption=Uruchomienie skryptu]
.\skrypt.ps1 
\end{lstlisting}
Wymagania:
\begin{itemize}
    \item system Windows
    \item OpenSSL w systemowym PATH
    \item Windows SDK (dla signtool.exe) \\ \\
\end{itemize}

\subsection{Działanie skryptu}

\begin{figure}[H]
    \centering
    \includegraphics[width=0.9\textwidth]{screenshots/skrypt_1} % (przykładowy zrzut wykonania skryptu)
    \caption{Wywołanie skryptu}
    \label{fig:skrypt_exec}
\end{figure}

\begin{figure}[H]
    \centering
    \includegraphics[width=0.9\textwidth]{screenshots/skrypt_2} % (przykładowy zrzut wykonania skryptu)
    \caption{Wywołanie skryptu}
    \label{fig:skrypt_exec}
\end{figure}

\subsection{Bezpieczeństwo}
Skrypt implementuje dobre praktyki bezpieczeństwa:
\begin{itemize}
    \item Wymagane uruchomienie jako administrator
    \item Weryfikacja obecności wymaganych narzędzi (OpenSSL, signtool)
    \item Zabezpieczenie kluczy hasłami
    \item Walidacja każdego etapu procesu
\end{itemize}

\subsection{Korzyści z automatyzacji}
\begin{itemize}
    \item Znaczne skrócenie czasu procesu
    \item Eliminacja błędów manualnych \\
\end{itemize}




\section{Podsumowanie projektu}

\noindent Projekt kompleksowo zrealizował proces podpisywania kodu w środowisku Windows, obejmujący:

\subsection*{Osiągnięcia}
\begin{itemize}
    \item \textbf{Infrastruktura certyfikacji}
    \begin{itemize}
        \item Utworzenie własnego Urzędu Certyfikacji (CA) z rozszerzeniami X.509v3
        \item Prawidłowa integracja z magazynem zaufanych certyfikatów Windows
        \item Generacja certyfikatów aplikacji z wymaganymi rozszerzeniami do podpisywania kodu
    \end{itemize}

    \item \textbf{Proces podpisywania}
    \begin{itemize}
        \item Konwersja certyfikatów do formatu PFX gotowego do użycia w IDE
        \item Implementacja podpisu cyfrowego w Visual Studio
        \item Ręczna procedura podpisywania z wykorzystaniem \texttt{signtool.exe}
    \end{itemize}

    \item \textbf{Zabezpieczenia czasowe}
    \begin{itemize}
        \item Integracja z publicznym serwerem TSA (DigiCert)
        \item Wdrożenie mechanizmu timestampingu SHA256
        \item Weryfikacja długoterminowej ważności podpisów
    \end{itemize}

    \item \textbf{Automatyzacja procesu}
    \begin{itemize}
        \item Opracowanie skryptu PowerShell automatyzującego wszystkie etapy
        \item Skrócenie czasu wykonania
    \end{itemize}
\end{itemize}

\subsection*{Potwierdzone efekty}
\begin{itemize}
    \item Poprawna weryfikacja podpisu w systemie Windows
    \item Brak ostrzeżeń o "nieznanym wydawcy"
    \item Powtarzalność wyników dzięki zautomatyzowanemu procesowi
\end{itemize}

\subsection*{Wnioski}
Projekt udowodnił możliwość budowy kompletnego rozwiązania do podpisywania kodu - od generacji infrastruktury kluczy, przez podpisywanie aplikacji, po weryfikację i automatyzację procesu.

\end{document}
